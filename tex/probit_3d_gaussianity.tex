\documentclass[11pt]{article}
\usepackage{amsmath,amssymb,amsthm}
\usepackage[margin=1in]{geometry}
\usepackage{graphicx}
\usepackage{booktabs}
\usepackage{float}

\newcommand{\E}{\mathbb{E}}
\newcommand{\Var}{\mathrm{Var}}
\newcommand{\Cov}{\mathrm{Cov}}
\newcommand{\kap}{\kappa}

\title{Gaussian Convergence in the 3D Probit Bandit:\\
Analysis via Sufficient Statistics $(Y, A, G)$}
\author{}
\date{}

\begin{document}
\maketitle

\section{Introduction}

We extend the analysis of late-time Gaussian convergence from the 1D case
(using $\gamma = \sum_t a_t y_t$) to a 3D sufficient statistic formulation.
The key finding is that \textbf{the 3D joint distribution also converges to
Gaussian as $t \to \infty$}, confirming that the Central Limit Theorem
intuition applies in this more general setting.

\section{Problem Setup}

\subsection{Sufficient Statistics}

Define three sufficient statistics:
\begin{align}
  Y &:= \sum_{t=1}^{T} y_t \quad \text{(total reward)} \\
  A &:= \sum_{t=1}^{T} a_t \quad \text{(action imbalance)} \\
  G &:= \sum_{t=1}^{T} a_t y_t \quad \text{(reward-action correlation)}
\end{align}

Note that $G$ is our previous $\gamma$. The tuple $(Y, A, G)$ provides a
3D state space for the bandit dynamics.

\subsection{Decision Variable}

The agent estimates arm means as:
\begin{align}
  \hat{\mu}_+ &= \frac{\sum_t y_t [a_t = +1]}{\sum_t [a_t = +1]}
             = \frac{Y + G}{T + A} \\
  \hat{\mu}_- &= \frac{\sum_t y_t [a_t = -1]}{\sum_t [a_t = -1]}
             = \frac{Y - G}{T - A}
\end{align}

The decision variable is:
\begin{equation}
  \hat{\mu}_+ - \hat{\mu}_- = \frac{Y + G}{T + A} - \frac{Y - G}{T - A}
  = \frac{2(GT - AY)}{T^2 - A^2}
\end{equation}

\subsection{Policy}

The probit policy is:
\begin{equation}
  P(a = +1 \mid Y, A, G, T) = \Phi\left(\beta \cdot \frac{GT - AY}{T^2 - A^2}\right)
\end{equation}

where $\Phi$ is the standard normal CDF and $\beta$ is the inverse temperature.

\section{Measures of Non-Gaussianity}

\subsection{Third-Order Cumulants (Skewness)}

For a 3D distribution $p(Y, A, G)$, there are 10 unique third-order
standardized cumulants:
\begin{itemize}
  \item \textbf{Marginal skewnesses:} $\kap_{300}/\sigma_Y^3$,
        $\kap_{030}/\sigma_A^3$, $\kap_{003}/\sigma_G^3$
  \item \textbf{Mixed third-order:} $\kap_{210}$, $\kap_{201}$, $\kap_{120}$,
        $\kap_{102}$, $\kap_{021}$, $\kap_{012}$
  \item \textbf{Co-skewness:} $\kap_{111}/(\sigma_Y \sigma_A \sigma_G)$
\end{itemize}

For a Gaussian distribution, all third-order cumulants are exactly zero.

\subsection{Fourth-Order Cumulants (Kurtosis)}

The marginal excess kurtoses are:
\begin{equation}
  \text{Excess kurtosis}(X) = \frac{\kap_4(X)}{\sigma_X^4} - 3
\end{equation}

For a Gaussian, excess kurtosis is zero for all marginals.

\section{Methods}

\subsection{Master Equation}

The exact probability distribution evolves via:
\begin{align}
  p(Y', A', G', t+1) = \sum_{Y,A,G} W(Y',A',G' \mid Y,A,G,t) \cdot p(Y,A,G,t)
\end{align}

where the transition kernel $W$ has four branches:
\begin{center}
\begin{tabular}{llll}
\toprule
Action & Reward & Transition & Probability \\
\midrule
$a=+1$ & $y=1$ & $(Y,A,G) \to (Y+1,A+1,G+1)$ & $P_+ \cdot \eta_+$ \\
$a=+1$ & $y=0$ & $(Y,A,G) \to (Y,A+1,G)$ & $P_+ \cdot (1-\eta_+)$ \\
$a=-1$ & $y=1$ & $(Y,A,G) \to (Y+1,A-1,G-1)$ & $P_- \cdot \eta_-$ \\
$a=-1$ & $y=0$ & $(Y,A,G) \to (Y,A-1,G)$ & $P_- \cdot (1-\eta_-)$ \\
\bottomrule
\end{tabular}
\end{center}

The state space grows as $O(T^2)$, limiting the master equation to $T \lesssim 100$.

\subsection{Monte Carlo Simulation}

We simulate $N = 5000$ independent trajectories and compute empirical moments
at each time step.

\section{Numerical Results}

\subsection{Parameters}

\begin{center}
\begin{tabular}{ll}
\toprule
Parameter & Value \\
\midrule
$m_+, m_-$ (prior pseudo-counts) & 10, 10 \\
$\beta$ (inverse temperature) & 0.5 \\
$\eta_+$ (arm 1 reward probability) & 0.6 \\
$\eta_-$ (arm 2 reward probability) & 0.4 \\
Time horizon $T$ & 100 \\
Monte Carlo samples & 5000 \\
\bottomrule
\end{tabular}
\end{center}

\subsection{Marginal Skewness Convergence}

Figure~\ref{fig:skewness} shows the marginal skewnesses for $Y$, $A$, and $G$
as functions of time. All three decay toward zero, indicating convergence to
Gaussian marginals.

\begin{figure}[H]
  \centering
  \includegraphics[width=0.95\textwidth]{../fig/probit_3d_gaussianity.png}
  \caption{Convergence to Gaussianity in the 3D probit bandit.
  \textbf{Top row:} Marginal skewnesses for $Y$, $A$, $G$ decay toward zero.
  \textbf{Bottom row:} Marginal excess kurtoses decay toward zero.
  Blue solid lines: Monte Carlo estimates. Red dashed lines: Master equation (exact).
  Both methods agree well.}
  \label{fig:skewness}
\end{figure}

\subsection{Co-Skewness Behavior}

The co-skewness $\kap_{YAG}/(\sigma_Y \sigma_A \sigma_G)$ measures the
three-way correlation structure. Interestingly:

\begin{itemize}
  \item Co-skewness starts at zero (initial delta distribution is trivially Gaussian)
  \item It peaks around $t \approx 10$ at approximately $0.25$
  \item It decays more slowly than marginal skewnesses
  \item At $t = 100$: co-skewness $\approx 0.10$
\end{itemize}

This suggests that cross-correlation non-Gaussianity persists longer than
marginal non-Gaussianity.

\begin{figure}[H]
  \centering
  \includegraphics[width=0.7\textwidth]{../fig/probit_3d_coskewness.png}
  \caption{Co-skewness $\kap_{YAG}$ evolution. The three-way correlation
  peaks during the transient and decays more slowly than marginal skewnesses.}
  \label{fig:coskew}
\end{figure}

\subsection{Summary Table}

\begin{center}
\begin{tabular}{ccccccc}
\toprule
$t$ & Skew($Y$) & Skew($A$) & Skew($G$) & Co-skew & Kurt($Y$) & Kurt($G$) \\
\midrule
0   & 0.00  & 0.00  & 0.00  & 0.00  & $-3.00$ & $-3.00$ \\
10  & 0.01  & $-0.05$ & 0.01  & 0.25  & $-0.26$ & $-0.23$ \\
25  & 0.01  & 0.01  & 0.04  & 0.18  & $-0.08$ & $-0.07$ \\
50  & 0.02  & 0.00  & 0.00  & 0.12  & $-0.14$ & $-0.09$ \\
75  & $-0.01$ & $-0.01$ & $-0.03$ & 0.12  & $-0.06$ & $-0.02$ \\
100 & $-0.01$ & $-0.03$ & $-0.05$ & 0.10  & 0.03  & 0.08 \\
\bottomrule
\end{tabular}
\end{center}

Note: The $t=0$ kurtosis of $-3$ reflects the delta function initial condition
(zero variance makes the excess kurtosis formula ill-defined; we report the
limiting value).

\subsection{State Space Growth}

The master equation tracks the exact distribution but faces exponential
state space growth:

\begin{center}
\begin{tabular}{cc}
\toprule
Time $t$ & Number of States \\
\midrule
10 & 286 \\
20 & 1,771 \\
30 & 5,416 \\
50 & 20,892 \\
80 & 50,000 (pruned) \\
100 & 50,000 (pruned) \\
\bottomrule
\end{tabular}
\end{center}

Beyond $t \approx 80$, we prune low-probability states to maintain tractability.

\section{Comparison with 1D Case}

In the 1D analysis using only $\gamma$, we found:
\begin{itemize}
  \item Skewness decays as $t^{-1/2}$ at late times
  \item Excess kurtosis decays similarly
  \item $K=2$ truncation error reaches machine precision by $t \sim 500$
\end{itemize}

The 3D case shows similar behavior:
\begin{itemize}
  \item Marginal skewnesses and kurtoses decay toward zero
  \item The joint distribution approaches a 3D Gaussian
  \item However, co-skewness decays more slowly, suggesting the
        cross-correlation structure is the last to become Gaussian
\end{itemize}

\section{Physical Interpretation}

The convergence to Gaussianity can be understood via the Central Limit Theorem:

\begin{enumerate}
  \item At each step, $(Y, A, G)$ receives increments $(\Delta Y, \Delta A, \Delta G)$
  \item The increments depend on the current state through the policy
  \item As $t$ grows, the distribution spreads (variance $\sim t$)
  \item The effective inverse temperature $\tilde{\beta} \propto \beta/\sigma \to 0$
  \item With $\tilde{\beta} \to 0$, the policy becomes diffuse (50-50)
  \item Increments become approximately i.i.d.
  \item The CLT then implies Gaussian convergence
\end{enumerate}

This mechanism is identical to the 1D case, extended to 3D.

\section{Conclusion}

\textbf{Main finding:} The 3D distribution $p(Y, A, G, t)$ converges to a
3D Gaussian as $t \to \infty$.

Quantitatively at $t = 100$:
\begin{itemize}
  \item Max $|$skewness$|$: 0.05 (approaching 0)
  \item Max $|$excess kurtosis$|$: 0.08 (approaching 0)
  \item Co-skewness: 0.10 (decaying more slowly)
\end{itemize}

This confirms that the late-time Gaussian approximation is valid for the
3D sufficient statistic formulation, just as it was for the 1D $\gamma$ case.
The co-skewness decay rate suggests that a Gaussian approximation becomes
accurate for marginals before the full joint distribution.

\end{document}
