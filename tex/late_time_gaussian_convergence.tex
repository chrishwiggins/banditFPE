\documentclass[11pt]{article}
\usepackage{amsmath,amssymb,amsthm}
\usepackage[margin=1in]{geometry}
\usepackage{graphicx}
\usepackage{booktabs}

\newcommand{\E}{\mathbb{E}}
\newcommand{\Var}{\mathrm{Var}}
\newcommand{\kap}{\kappa}

\title{Late-Time Gaussian Convergence and\\$K=2$ Truncation Accuracy}
\author{}
\date{}

\begin{document}
\maketitle

\section{Summary}

We verify numerically that for the $\beta$-$\Gamma$ bandit, the probability distribution
$p(\gamma, t)$ converges to a Gaussian as $t \to \infty$. Consequently, the $K=2$
(Gaussian) truncation of the Edgeworth expansion becomes increasingly accurate at late times.

\section{Physical Intuition}

The Central Limit Theorem (CLT) applies to the accumulated log-likelihood ratio $\gamma$
because:
\begin{enumerate}
  \item At each time step, $\gamma$ receives an increment $\xi = ya \in \{+1, -1\}$.
  \item While increments are not independent (they depend on $\gamma$ through the policy),
        the variance grows as $\sigma^2 \sim t$, spreading the distribution.
  \item As $\sigma \to \infty$, the effective inverse temperature
        $\tilde{\beta} = \beta/\sqrt{1 + \beta^2\sigma^2} \to 0$.
  \item With $\tilde{\beta} \to 0$, the policy becomes diffuse ($b \to 0$),
        and increments become approximately i.i.d.
\end{enumerate}

Thus, at late times, the distribution approaches Gaussian, and higher cumulants become
negligible.

\section{Standardized Cumulants}

The proper measures of non-Gaussianity are the \textbf{standardized cumulants}:
\begin{align}
  \text{Skewness} &= \frac{\kap_3}{\sigma^3}, \\
  \text{Excess kurtosis} &= \frac{\kap_4}{\sigma^4}.
\end{align}

For a Gaussian distribution, both are exactly zero. As $t \to \infty$, we expect:
\begin{equation}
  \frac{\kap_j}{\sigma^j} \sim t^{-(j-2)/2}, \quad j \geq 3.
\end{equation}

Note that the \emph{raw} cumulants $\kap_3, \kap_4$ grow with time (since $\sigma$ grows),
but the standardized versions decay.

\section{Truncation Rate Error}

The truncation rate error measures how well the $K$-truncated Edgeworth ansatz predicts
the instantaneous cumulant update:
\begin{equation}
  \epsilon_j^K(t) = |\Delta\kap_j^{\text{exact}}(t) - \Delta\kap_j^{K\text{-ansatz}}(t)|,
\end{equation}
where:
\begin{itemize}
  \item $\Delta\kap_j^{\text{exact}} = \kap_j(t+1) - \kap_j(t)$ from the exact master equation.
  \item $\Delta\kap_j^{K\text{-ansatz}}$ is computed from the $K$-truncated Edgeworth formulas,
        evaluated with the \emph{exact} cumulants $\kap_1(t), \ldots, \kap_K(t)$.
\end{itemize}

This is not trajectory error (which accumulates), but rather measures how accurately
the truncated ansatz predicts the next step given the true current state.

\section{Numerical Results}

\subsection{Parameters}

\begin{center}
\begin{tabular}{ll}
\toprule
Parameter & Value \\
\midrule
$m_+, m_-$ (initial counts) & 10, 10 \\
$\beta$ (inverse temperature) & 0.3 \\
$\eta_+$ (arm 1 reward rate) & 0.6 \\
$\eta_-$ (arm 2 reward rate) & 0.4 \\
$\bar{\eta} = (\eta_+ + \eta_-)/2$ & 0.5 \\
$\Delta\eta = (\eta_+ - \eta_-)/2$ & 0.1 \\
Time horizon $T$ & 1000 \\
\bottomrule
\end{tabular}
\end{center}

\subsection{Late-Time Convergence}

Figure~\ref{fig:convergence} shows four key results:

\begin{figure}[htbp]
  \centering
  \includegraphics[width=0.95\textwidth]{../fig/late_time_gaussian_convergence.png}
  \caption{Late-time Gaussian convergence. \textbf{Top left:} $K=2$ truncation rate error
  for mean and variance updates decays to machine precision.
  \textbf{Top right:} Standardized cumulants (skewness, excess kurtosis) decay toward zero.
  \textbf{Bottom left:} Error vs truncation order $K$ at various times.
  \textbf{Bottom right:} Power-law decay of truncation error on log-log scale.}
  \label{fig:convergence}
\end{figure}

\begin{enumerate}
  \item \textbf{Truncation error decay (top left):} The $K=2$ closure errors
        $\epsilon_1^{K=2}$ and $\epsilon_2^{K=2}$ peak during the transient
        ($t \sim 50$) at $O(10^{-4})$ and $O(10^{-2})$ respectively, then
        decay to machine precision ($O(10^{-14})$) by $t \sim 500$.

  \item \textbf{Standardized cumulant decay (top right):} Skewness decays from
        peak values ($\sim -0.45$ at $t=50$) to $-0.06$ at $t=1000$; excess kurtosis
        decays from $0.56$ to $0.01$. Both approach zero as $t \to \infty$,
        confirming Gaussian convergence.

  \item \textbf{Exact vs ansatz rates (bottom left):} The $K=2$ ansatz
        accurately predicts the variance rate $\Delta\kap_2$ across all times,
        with agreement improving as the distribution becomes Gaussian.

  \item \textbf{Non-Gaussianity measures (bottom right):} Skewness and excess
        kurtosis decay monotonically toward zero, directly demonstrating
        convergence to a Gaussian distribution.
\end{enumerate}

\subsection{Summary Table}

\begin{center}
\begin{tabular}{cccc}
\toprule
Time $t$ & Skewness & Excess Kurtosis & $\epsilon_2^{K=2}$ \\
\midrule
0    & $0.00$   & $0.00$  & $4 \times 10^{-15}$ \\
50   & $-0.45$  & $0.56$  & $4 \times 10^{-2}$ \\
100  & $-0.32$  & $0.39$  & $6 \times 10^{-4}$ \\
200  & $-0.19$  & $0.15$  & $8 \times 10^{-8}$ \\
500  & $-0.09$  & $0.03$  & $6 \times 10^{-14}$ \\
1000 & $-0.06$  & $0.01$  & $1 \times 10^{-13}$ \\
\bottomrule
\end{tabular}
\end{center}

By $t \sim 500$, the truncation error reaches machine precision, confirming that
the distribution is effectively Gaussian and the $K=2$ closure is exact.

\section{Rates Comparison}

Figure~\ref{fig:rates} compares the exact cumulant rates $\Delta\kap_j^{\text{exact}}$
with the ansatz predictions for various truncation orders.

\begin{figure}[htbp]
  \centering
  \includegraphics[width=0.95\textwidth]{../fig/rates_comparison.png}
  \caption{Comparison of exact cumulant rates vs $K$-truncated ansatz predictions.
  \textbf{Top left:} Mean rate $\Delta\kap_1$.
  \textbf{Top right:} Variance rate $\Delta\kap_2$.
  \textbf{Bottom left:} Exact rates for $\kap_1$ through $\kap_4$.
  \textbf{Bottom right:} Relative error in variance rate.}
  \label{fig:rates}
\end{figure}

\section{Truncation Rate Error Detail}

Figure~\ref{fig:truncation} provides detailed analysis of the truncation rate error
for varying $K$ and $j$.

\begin{figure}[htbp]
  \centering
  \includegraphics[width=0.95\textwidth]{../fig/truncation_rate_error.png}
  \caption{Truncation rate error over $T=1000$ time steps.
  \textbf{Top left:} $K=2$ error in mean rate peaks during transient then decays.
  \textbf{Top right:} $K=2$ error in variance rate follows similar pattern.
  \textbf{Bottom left:} Error for $j=1,2$ with fixed $K=2$.
  \textbf{Bottom right:} Non-Gaussianity measures (skewness, excess kurtosis) decay toward zero.}
  \label{fig:truncation}
\end{figure}

\section{Conclusion}

The numerical results confirm the physical intuition:

\begin{quote}
\emph{Post-transient, $p(\gamma, t)$ becomes Gaussian as $t \to \infty$,
therefore the $K=2$ Edgeworth truncation becomes exact in the late-time limit.}
\end{quote}

Quantitatively at $T=1000$:
\begin{itemize}
  \item Skewness: $-0.06 \to 0$ as $t \to \infty$
  \item Excess kurtosis: $0.01 \to 0$ as $t \to \infty$
  \item $K=2$ truncation error: $O(10^{-14})$ (machine precision) by $t \sim 500$
\end{itemize}

For practical purposes, the $K=2$ (Gaussian) closure is sufficient for late-time
dynamics. Higher-order closures ($K \geq 3$) are only necessary during the transient
regime ($t \lesssim 100$) when the distribution has significant non-Gaussian features.

\end{document}
